Inverse problem states for the process of retrieving causal factors that best explain observations 
of a phenomenon. In simulated physics, it opposes to forward problem which starts from causes
to produce consequences. Its simplest form being, from some initial conditions, simulate a system (of ODE, SDE, ...) 
whereas trying to find best initial conditions that have produced this simulation
is an inverse problem.\\

This can take several forms. One could estimate states of a phenomenon as well as factors
that characterize it, ...
For example ... citer des domaines\\

Data assimilation is a specific type of inverse problem in which states of a system are estimated
from observations. The purpose being to assimilate a window of fresh observations to make predictions 
about plausible states that have generated them. By doing so, one could start from this state to make
a forecast.\\ 
This is intensively used in weather predictions. Starting from a prior knowledge about the
phenomenon, one can simulate for the atmospheric state evolution in order to predict the weather in the 
coming few days. Nevertheless, this only holds for perfect simulators and infinitely precise states.
Indeed, such big systems are very chaotic and simulators are often an approximation of the reality.
A solution being to frequently refresh the estimation of the current state by taking into account
observations and make the simulated trajectory better fit the reality. ECMWF uses 4DVar ... speak about it

% In the context of simulators, it is the process which associates plausible simulatorparameters to 
% observations (e.g. initial state, )